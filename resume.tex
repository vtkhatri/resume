%-------------------------
% Resume in Latex
% Author : Viraj Khatri
% Based on: https://github.com/sb2nov/resume and https://github.com/samuelgonsalves/resume
%------------------------

\documentclass[letterpaper,11pt]{article}

\usepackage{latexsym}
\usepackage[empty]{fullpage}
\usepackage{titlesec}
\usepackage{marvosym}
\usepackage[usenames,dvipsnames]{color}
\usepackage{verbatim}
\usepackage{enumitem}
\usepackage[hidelinks]{hyperref}
\usepackage{fancyhdr}
\usepackage[english]{babel}
\usepackage{tabularx}
\input{glyphtounicode}

\usepackage{pagecolor}
\usepackage{lato}
\renewcommand{\familydefault}{\sfdefault}
\usepackage{fontawesome}

\pagestyle{fancy}
\fancyhf{} % clear all header and footer fields
\fancyfoot{}
\renewcommand{\headrulewidth}{0pt}
\renewcommand{\footrulewidth}{0pt}

% Adjust margins
\addtolength{\oddsidemargin}{-0.5in}
\addtolength{\evensidemargin}{-0.5in}
\addtolength{\textwidth}{1in}
\addtolength{\topmargin}{-.5in}
\addtolength{\textheight}{1.0in}

\urlstyle{same}

\raggedbottom
\raggedright
\setlength{\tabcolsep}{0in}

% Sections formatting
\titleformat{\section}{
  \vspace{-4pt}\scshape\raggedright\large
}{}{0em}{}[\color{black}\titlerule \vspace{-5pt}]

% Ensure that generate pdf is machine readable/ATS parsable
\pdfgentounicode=1

%-------------------------
% Custom commands
\newcommand{\resumeItem}[2]{
  \item\small{
    \textbf{#1}{: #2 \vspace{-2pt}}
  }
}

% Just in case someone needs a heading that does not need to be in a list
\newcommand{\resumeHeading}[4]{
    \begin{tabular*}{0.99\textwidth}[t]{l@{\extracolsep{\fill}}r}
      \textbf{#1} & #2 \\
      \textit{\small#3} & \textit{\small #4} \\
    \end{tabular*}\vspace{-5pt}
}

\newcommand{\resumeSubheading}[4]{
  \vspace{-1pt}\item
    \begin{tabular*}{0.97\textwidth}[t]{l@{\extracolsep{\fill}}r}
      \textbf{#1} & #2 \\
      \textit{\small#3} & \textit{\small #4} \\
    \end{tabular*}\vspace{-5pt}
}

\newcommand{\resumeSubSubheading}[2]{
    \begin{tabular*}{0.97\textwidth}{l@{\extracolsep{\fill}}r}
      \textit{\small#1} & \textit{\small #2} \\
    \end{tabular*}\vspace{-5pt}
}

\newcommand{\resumeSubItem}[2]{\resumeItem{#1}{#2}\vspace{-4pt}}

\renewcommand{\labelitemii}{$\circ$}

\newcommand{\resumeSubHeadingListStart}{\begin{itemize}[leftmargin=*]}
\newcommand{\resumeSubHeadingListEnd}{\end{itemize}}
\newcommand{\resumeItemListStart}{\begin{itemize}}
\newcommand{\resumeItemListEnd}{\end{itemize}\vspace{-5pt}}

% COLOR THEME
% For Light theme un-comment this and comment the Dark theme section below
%\colorlet{urlcolor}{blue}
%\newcommand{\otherThemeRef}{\href{https://github.com/wilmeragsgh/resume/raw/master/wilmer_gonzalez_dark.pdf}{See dark theme}}
%\newcommand{\latestVersion}{\href{https://github.com/wilmeragsgh/resume/raw/master/wilmer_gonzalez_light.pdf}{Get Latest version}}

% For Dark theme un-comment this and comment the Light theme section above

\colorlet{textcolor}{white!80!gray}
\colorlet{backgroundcolor}{black!30!gray}
\colorlet{urlcolor}{blue!25!white}
\AtBeginDocument{\color{textcolor}}
%\newcommand{\otherThemeRef}{\href{https://github.com/wilmeragsgh/resume/raw/master/wilmer_gonzalez_light.pdf}{See light theme}}
%\newcommand{\latestVersion}{\href{https://github.com/wilmeragsgh/resume/raw/master/wilmer_gonzalez_dark.pdf}{Get Latest version}}
% ---

% DOCUMENT MAIN
\begin{document}

% For Dark theme also un-comment this line:
\pagecolor{backgroundcolor}

% HEADING
\begin{tabular*}{\textwidth}{l@{\extracolsep{\fill}}r}
	\textbf{\Large Viraj Khatri} &  \href{https://github.com/vtkhatri}{ \faicon{github} \color{urlcolor} vtkhatri} \\
	\href{mailto:virajkhatri@gmail.com}{virajkhatri@gmail.com} & 
	\href{tel:14699224573}{+1 469 922 4573}
	%&  \faicon{code} Python, R\\
	%\textsl{\small \latestVersion} & \textsl{\small \otherThemeRef}
\end{tabular*}
% ---

% OBJECTIVE
\section{Objective}
Graduate Architecture Intern at Hillsboro or Santa Clara

% EDUCATION
\section{Education}
	\resumeSubHeadingListStart
		\resumeSubheading
			{Portland State University}{Portland, OR}
			{M.Sc Electrical and Computer Engineering, - / 4.0}{Fall 2021 -- Spring 2023}
			\resumeItemListStart
				\resumeItem{Relevant Courses}{}
				\resumeItemListStart
					\resumeItem {ECE 581}{ASIC Modelling and Synthesis (Dr. Xaiyou Song - Fall 2021}
					\resumeItem {ECE 585}{Microprocessor System Design (Mark Faust) - Fall 2021}
					\resumeItem {(tentative) ECE 586 }{Computer Architecture (Mark Faust) - Winter 2022}
					\resumeItem {(tentative) ECE 587}{Advanced CompArch I (Yuchen Huang) - Spring 2022}
					\resumeItem {(tentative) ECE 588}{Advanced CompArch II (Yuchen Huang) - Fall 2022}
				\resumeItemListEnd
			\resumeItemListEnd
		\resumeSubheading
			{College of Engineering, Pune}{Pune, Maharashtra, India}
			{B.Tech Electronics and Telecommunication Engineering, 3.2 / 4.0 (7.99/10)}{May 2015 -- June 2019}
	\resumeSubHeadingListEnd
% ---

% EXPERIENCE
\section{Experience}
	\resumeSubHeadingListStart
		\resumeSubheading
			{Tejas Networks}{Mumbai, Maharashtra, India}
			{Research and Development Engineer}{August 2019 - August 2021}
			\resumeItemListStart
				\resumeItem{Networking Technologies}{Continuously working with DHCP, VLAN tagged Traffic management, VPNs, Downstream Ingress Bandwidth, HQOS queuing, Traffic Shaping Profiles, etc.}
				\resumeItem{C/C++}{The software used for configuring switching capabilities of a network card is largely C for device drivers and C++ for higher level UI.}
				\resumeItem{Feature Development}{Implemented Zero Touch In-Band Management feature request by Tejas Network's client in collaboration with QA team.}
				\resumeItem{Python}{Extensive scripting to trivialize monotonous commands with flexibility to adapt to situations.}
				\resumeItem{Training}{Trained new recruits to the team and enable them to contribute meaningfully.}
				\resumeItem{SQL}{Configuration stored on the network cards is in a database, needs to be created , updated, or re-played onto the hardware are required.}
			\resumeItemListEnd

		\resumeSubheading
			{DOT Sys Technologies}{Mumbai, Maharashtra, India}
			{Design Intern}{May 2018 - August 2018}
			\resumeItemListStart
				\resumeItem{Transistor Theory}{Implemented Pulse Width Modulation to control voltage and current levels to make a within constrains of Transistor hardware.}
				\resumeItem{Power Electronics}{Used loose capacitors, inductions and Transformers to convert main lines supply to transistor switching compatible levels.}
				\resumeItem{Arduino}{Made a Programmable Battery Charger with UI implemented on Arduino + Transistor Theory and Power Electronics to manage the actual charging.}
			\resumeItemListEnd
     
		\resumeSubheading
			{Eduvance}{Mumabai, Maharashtra, India}
			{Intern (B.Tech. project - Smart Paper Tracking System)}{May 2017 - January 2020}
			\resumeItemListStart
				\resumeItem{IoT}{Used RPi to collect data from devices via Bluetooth and IBM cloud services to implement data storage, sync, and decision making on cloud}
				\resumeItem{Bluetooth}{Used Cypress Semiconductors PSOC4-BLE boards as portable markers to be attached to files to track them and provide information to RPi for syncing.}
			\resumeItemListEnd

	\resumeSubHeadingListEnd
% ---

% PROJECTS
\section{Projects}

	\resumeSubHeadingListStart
		\resumeSubItem{\href{https://github.com/vtkhatri/hamming_code}{hamming\_code}}{Asynchronous Hamming Encoder and Decoder in System Verilog.}
		\resumeSubItem{\href{https://github.com/vtkhatri/fifo}{fifo}}{Parameterized fifo with no handshaking protocols for outgoing communication.}
		\resumeSubItem{\href{https://github.com/vtkhatri/i2c}{i2c}}{ I2C master and slave implemented in Verilog, currently re-writing code in to System Verilog.}
		\resumeSubItem{\href{https://github.com/vtkhatri/ffind}{ffind}}{Small wrapper for find command in linux to make find accept grep-like arguments.}
		\resumeSubItem{\href{https://github.com/enricocid/Music-Player-GO}{Music-Player-GO}}{Feature Contributions to open source Music Player app.}
	\resumeSubHeadingListEnd
% ---

% PERSONAL PUBLICATIONS
\section{Personal publications}

	\resumeSubHeadingListStart
	\resumeSubItem
		{Modified MD5 Algorithm for Low-End IoT Edge Devices.}{Viraj Khatri, Dr. Vanita Agarwal. \href{https://doi.org/10.1109/ICCCNT45670.2019.8944533} {\color{urlcolor}ICCCNT2019}}
		%\resumeSubSubheading {Increasing amount of information has left a chasm of difference between low-end and high-end processor requirements. This can be addressed by scaling down the MD5 Algorithm. We present the performance characteristics of modified versions of MD5 message-digest algorithm which have been altered using two different approaches, reducing lateral register widths or reducing computation cycles. The metrics for comparison are collisions, synthesis reports and time required for message-digest calculation.}

	\resumeSubHeadingListEnd
% ---

% PROGRAMMING SKILLS
%\section{Other programming tools}
%	\resumeSubHeadingListStart
%	\resumeSubItem{\textbf{Neural Network Frameworks} Keras, Tensorflow.}
%	\resumeSubHeadingListEnd
% ---

\end{document}